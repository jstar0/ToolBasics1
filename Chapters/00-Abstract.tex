\thispagestyle{plain} % Page style without header and footer
\pdfbookmark[1]{概述}{概述} % Add entry to PDF
\chapter*{概述} % Chapter* to appear without numeration
\label{cp:abstract}

本实验报告使用由\textbf{于景一}制作的\LaTeX{}模板完成。关于此模板的信息,您可以前往\href{https://github.com/jstar0/LaTeXTemplate/}{GitHub模板仓库}具体了解。\footnote{或您可直接搜索GitHub账号\textit{@jstar0}了解更多}\footnote{您请注意,本模板基于LPPL v1.3c分发,本项目在原模板\href{https://github.com/joseareia/ipleiria-thesis}{Polytechnic University of Leiria: LaTeX Thesis Template}的基础上进行了合法地大量二改,包括但不限于自定义风格、中文化支持、样式重定义、功能增加等。}\footnote{模板提供两种样式,一种为学术论文样式,另一种为实验报告样式,具体区别请检查GitHub仓库上的两个分支。}\\

本实验报告是\textit{系统开发工具基础课程}的第一次实验报告,主要关于\textbf{使用Git进行版本控制,以及掌握\LaTeX{}模板的基本使用方法},总结个人经验,记录心得体会。 \\

关于Git的部分,我们主要学习了Git的基本概念,并通过实际操作,掌握了它的基本使用方法,包括但不限于初始化仓库、添加文件、提交更改、查看历史、分支管理等。特别的,由于我们在\textit{程序设计基础实践}课程进行了小组协作(使用GitHub进行代码托管和版本管理),我们在实操中练习了使用Git进行远端操作、版本控制,以及如何解决冲突等问题。\\

至于\LaTeX{}部分,我在制作个人模板、完成\textit{计算机工程伦理}的课程报告\footnote{如果您对此感兴趣,可以点击\href{https://xinera-my.sharepoint.com/:b:/g/personal/i_mcxin_top/EZ0DMB5suB5Dgu74rP2IwnsBF8CHPOEvkMRq40TBv-3rSQ?e=dU1OTj}{此链接}进行预览。}的过程中,对\LaTeX{}有了更深入的了解。在实操中,我学习了\LaTeX{}的基本语法,包括但不限于文档结构、文本格式、数学公式、插图表格、参考文献等。例如本模板,就是学习的成果,使用交叉编译,采用\textit{XeLaTeX$\rightarrow$ BibTeX$\rightarrow$ XeLaTeX$\rightarrow$ XeLaTeX}的编译顺序,支持中文、英文混排,支持引用文献,兼具较好的兼容性和美观性。\\

\note{本模板仍非特别完善,仍在活跃维护中,接下来将会进一步提升本地化水平,欢迎您提出宝贵意见。}